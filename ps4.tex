\documentclass[12pt, letterpaper]{article}
\title{Problem Set 4: Forced and Damped Pendulum}
\author{Ken Sheedlo}
\usepackage[pdftex]{graphicx}
\usepackage{fullpage}
\usepackage{verbatim}
\usepackage{amsfonts}
\usepackage{caption}
\usepackage{subcaption}

\begin{document}
\maketitle{}

\section*{Undriven, Undamped System}

In order to better analyze the behavior of a chaotic pendulum system, it's 
important to understand the behavior of a simple pendulum, without damping or a 
drive. The first pendulum system we will examine was created with $m=0.1$ kg, 
$l=0.1$ m, $\beta=0$ and without drive. Figure 1 shows a plot of the state-space
trajectory starting from the point $[\theta, \omega] = [3, 0.1]$. 

\begin{center}
\includegraphics[scale=0.6]{ps4_2a.png}
\\
Figure 1: State-space trajectory in an undamped, undriven pendulum.
\end{center}

This trajectory is analogous to picking the pendulum almost, but not quite 
straight up standing vertically, and then dropping it. In an undamped and 
undriven system, the pendulum returns to the same height at which it was dropped
and comes back, oscillating forever. The starting point at $[3, 0.1]$ is close
to an unstable equilibrium point at $[\pi, 0]$. The system will not move from
that point on its own, but will move and not return there if it is perturbed.

A second state-space trajectory was generated starting from 
$[\theta, \omega] = [0.01, 0]$. The results are shown in Figure 2.

\begin{center}
\includegraphics[scale=0.6]{ps4_2b.png}
\\
Figure 2: State-space trajectory from $[\theta, \omega] = [0.01, 0]$.
\end{center}

This trajectory is much smaller and more perfectly elliptical than the 
trajectory shown in Figure 1, which is large and shaped more like a football or
a diamond.

\section*{State-Space Portraits}

With the same parameters to the pendulum as in the examples above, I generated a
state-space portrait of the system. Figure 3 shows the resulting state-space 
portrait for the undamped, undriven pendulum.

\begin{center}
\includegraphics[scale=0.6]{ps4_3.png}
\\
Figure 3: State-space portrait for an undamped, undriven pendulum.
\end{center}

The elliptical orbits near the center represent a pendulum swinging through a 
small area near the bottom. Football-shaped orbits further out reflect the 
behavior when the pendulum swings near the top but does not go over, and the 
higher wave-shaped orbits reflect behavior when the pendulum whirls over the 
top.

Figure 4 shows a state-space portrait of a damped pendulum with $m=0.1$ kg, 
$l=0.1$ m, $\beta=0.25$ and without drive. 

\begin{center}
\includegraphics[scale=0.6]{ps4_4.png}
\\
Figure 4: State-space portrait for a damped pendulum.
\end{center}

This plot demonstrates the effect of the damping mechanic on the system. 
Trajectories spiral into the center where they would have looped forever or 
whirled over the top in the undamped system. Of course, the center represents 
the state where the pendulum is stopped pointing straight down. This is a stable
fixed point at $[\theta, \omega] = [0, 0]$. 

Plotting the phase portrait with $\theta$ modulo $2\pi$ brings attention to the
unstable fixed points of the system. An example is shown in Figure 5. 

\begin{center}
\includegraphics[scale=0.6]{ps4_5.png}
\\
Figure 5: State-space portrait for a damped pendulum ($\theta$ mod $2\pi$).
\end{center}

Drawing the plot with $\theta$ modulo $2\pi$ meant I had to use dot markers 
instead of smooth curves. Figure 5 resembles Figure 4 if it had been shifted 
$\pi$ to the left. An unstable fixed point is now in the center. This point
corresponds to a pendulum pointing straight up and not falling in either 
direction, but standing still.

\section*{Chaotic Trajectories}

I found a chaotic orbit with parameters $m=0.1$, $l=0.1$, $\beta=0.25$, $A=1$,
and $\alpha=7.4246$. Figure 6 plots the state-space trajectory over 100,000 
steps of fourth-order Runge-Kutta.

\begin{center}
\includegraphics[scale=0.6]{ps4_6.png}
\\
Figure 6: Chaotic pendulum orbit.
\end{center}

\section*{Numerical Effects}

For experimenting with the timestep, I used the same parameters as in Figure 1. 
Increasing the timestep $h$ gave a progression of degenerate results. For 
$h<0.03$, there was little noticeable effect. For $0.03 < h < 0.07$, the 
football shape from Figure 1 turns into a fat oval with a large hole in the 
center. For $0.07 < h < 0.2$, the trajectory spirals into the center at 
$[\theta, \omega] = [0,0]$. These effects could be explained by the algorithm
looking slightly too far ahead and detecting the curve sweeping around, 
overcorrecting for itself. For $h > 0.3$, the trajectory escapes the basin 
around the origin and diverges wildly in an apparently random direction. This 
could be explained by the algorithm reaching outside the basin around the origin
to measure the derivative and adding it together with measurements from inside
the basin, generating unpredictable results.

\end{document}