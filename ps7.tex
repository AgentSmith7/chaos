\documentclass[12pt, letterpaper]{article}
\title{Problem Set 7: Variational Equation}
\author{Ken Sheedlo}
\usepackage[pdftex]{graphicx}
\usepackage{fullpage}
\usepackage{verbatim}
\usepackage{amsfonts}
\usepackage{caption}
\usepackage{subcaption}

\begin{document}
\maketitle{}

\section*{1. Lorenz System Jacobian}

We wish to find the Jacobian of the Lorenz System. Recall that the Jacobian is
written as 

\begin{equation}
D_x\vec{F} = \left[ \begin{array}{ccc}
\frac{\partial f_x}{\partial x} & \frac{\partial f_x}{\partial y} & 
\frac{\partial f_x}{\partial z} \\ \frac{\partial f_y}{\partial x} & 
\frac{\partial f_y}{\partial y} & \frac{\partial f_y}{\partial z} \\
\frac{\partial f_z}{\partial x} & \frac{\partial f_z}{\partial y} & 
\frac{\partial f_z}{\partial z} \\
\end{array} \right]
\end{equation}

and the Lorenz system is

\begin{equation}
\vec{F}(\vec{x}, a, r, b) = \left[ \begin{array}{c}
\dot{x} \\ \dot{y} \\ \dot{z}
\end{array} \right] = \left[ \begin{array}{c}
a(y-x) \\ rx - y - xz \\ xy - bz
\end{array} \right]
\end{equation}

Taking partials of $\dot{x}$, $\dot{y}$ and $\dot{z}$ with respect to $x$, $y$
and $z$ yields the following system of equations:

\begin{equation}
\frac{\partial f_x}{\partial x} = -a
\end{equation}

\begin{equation}
\frac{\partial f_x}{\partial y} = a
\end{equation}

\begin{equation}
\frac{\partial f_x}{\partial z} = 0
\end{equation}

\begin{equation}
\frac{\partial f_y}{\partial x} = r
\end{equation}

\begin{equation}
\frac{\partial f_y}{\partial y} = -1
\end{equation}

\begin{equation}
\frac{\partial f_y}{\partial z} = -x
\end{equation}

\begin{equation}
\frac{\partial f_z}{\partial x} = y
\end{equation}

\begin{equation}
\frac{\partial f_z}{\partial y} = x
\end{equation}

\begin{equation}
\frac{\partial f_z}{\partial z} = -b
\end{equation}

Written as a Jacobian, these equations give

\begin{equation}
D_x\vec{F} = \left[ \begin{array}{ccc}
-a & a & 0 \\
r & -1 & -x \\
y & x & -b 
\end{array} \right]
\end{equation}

\section*{2. Lorenz Variational System}

The variational system for the Lorenz equation can be written as $\dot{\delta}
= D_x\vec{F}\delta$. In terms of the Jacobian matrix and the matrix of variations,
this is

\begin{equation}
\dot{\delta} = \left[ \begin{array}{ccc}
-a & a & 0 \\
r & -1 & -x \\
y & x & -b
\end{array} \right] \left[ \begin{array}{ccc}
\delta_{xx} & \delta_{yx} & \delta_{zx} \\
\delta_{xy} & \delta_{yy} & \delta_{zy} \\
\delta_{xz} & \delta_{yz} & \delta_{zz} 
\end{array} \right]
\end{equation}

So, $\dot{\delta}$ is a $3 \times 3$ matrix of derivatives, and $\delta_{xy}$ is
the component of the $x$-variation that comes from the previous $y$-variation. 
We can use the matrix of derivatives of variations to integrate the change in 
variations over the system.

\section*{3. Numerical Analysis}

Integrating the system from initial condition (a) as described in the writeup
gave the following results:

\begin{equation}
\delta = \left[ \begin{array}{ccc}
2.545755 & 1.980193 & -0.030017 \\
5.557742 & 4.395069 & -0.090683 \\
0.506627 & 0.384858 & 0.665060
\end{array} \right]
\end{equation}

\vspace{0.6em}

with column sums $\sum{\delta_x} = 8.61024$, $\sum{\delta_y} = 6.760119$, and 
$\sum{\delta_z} = 0.544360$. 

\vspace{1em}

Integrating the system from initial condition (b) gave the following results:

\begin{equation}
\delta = \left[ \begin{array}{ccc}
2.333913 & 1.751566 & -0.509134 \\
4.566961 & 3.461204 & -1.218821 \\
3.383342 & 2.669810 & -0.045495 
\end{array} \right]
\end{equation}

\vspace{0.6em}

with column sums $\sum{\delta_x} = 10.284215$, $\sum{\delta_y} = 7.882581$, and
$\sum{\delta_z} = -1.773450$. 

\vspace{1em}

Integrating the system from initial condition (c) gave the following results:

\begin{equation}
\delta = \left[ \begin{array}{ccc}
2.545755  &  1.980193  & 0.030017 \\
5.557742  &  4.395069  & 0.090683 \\
-0.506627 &  -0.384858 & 0.665060 
\end{array} \right]
\end{equation}

\vspace{0.6em}

with column sums $\sum{\delta_x} = 7.596870$, $\sum{\delta_y} = 5.990404$, and
$\sum{\delta_z} = 0.785760$. 

\vspace{1em}

The variations evidently grow fastest from point (b), where the $x$ component 
of the variation grows by a factor of 10, the $y$ by almost a factor of 8, and
the $z$ flips around backwards to $-1.7$. In fact, the $z$ component from point
(b) is interesting in that it is the only component from any of the initial
conditions tested that actually experiences negative growth. The $x$ and $y$ 
components consistently experience substantial positive growth from all initial
conditions. The $z$ components of evolved variations from points (a) and (c) are
both positive and less than 1. This means that they should be getting smaller 
but not changing sign.

\end{document}