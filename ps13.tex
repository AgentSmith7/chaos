\documentclass[12pt, letterpaper]{article}
\title{Problem Set 13: Three-Body Dynamics}
\author{Ken Sheedlo}
\usepackage[pdftex]{graphicx}
\usepackage[margin=1.0in]{geometry}
\usepackage{verbatim}
\usepackage{amsfonts}
\usepackage{caption}
\usepackage{subcaption}

\begin{document}
\maketitle{}

\section*{Three-Body Implementation and Testing}

Adding a third body to my equation from Problem Set 12 was relatively 
straightforward. To test the new system, I constructed the old two-body system
along with a field star 3000 units away along the positive $y$-axis. Figure 1
shows the results in a space-fixed reference frame as well as in a frame with
star A at the origin.

\begin{center}
\begin{minipage}[t]{0.49\textwidth}
\includegraphics[scale=0.5]{ps13_1a_cropped.png}
\end{minipage}
\begin{minipage}[t]{0.49\textwidth}
\includegraphics[scale=0.5]{ps13_1b_cropped.png}
\end{minipage}
\\
Figure 1: Orbital dynamics of a binary with a distant field star.
\end{center}

Note that the field star is not shown in the plot. This was done to better 
demonstrate the dynamics of the binary. The plots clearly show that the field
star 3000 units away has no noticeable effect on the dynamics of the binary. 

\section*{Three-Body Orbital Interaction}

For the first real example of a three-body orbital interaction, I set Stars A
and B in the same orientation and initial velocity as before, and placed the
field star C at $\left<0, 20, 0\right>$ with velocity $\left<0, -0.15,
0\right>$. Figure 2 plots the interaction in a space-fixed reference frame.

\begin{center}
\includegraphics[scale=0.6]{ps13_2a.png}
\\
Figure 2: Three-body stellar collision.
\end{center}

The stars appear to pass quite close together in this collision. Following the
collision, they are flung apart in different directions. The binary is separated
and no new capture occurs. This result does not resemble Hut and Bahcall's plot,
and it should not resemble their plot. Hut and Bahcall used a frame of reference
that was fixed to the center of mass of the system, while I used a reference
frame that is fixed in space. Further, Hut and Bahcall's result has star B
ejected from the binary while star C is captured, forming a new binary with star
A. So the results are topologically different.

\section*{Changing the True Anomaly}

Increasing the distance of the field star gradually, I found a number of
scenarios where star A is ejected and stars B and C form a new binary. I will
describe the distance of the field star as $x_{14}$ because the position in the
y direction of star C is the fourteenth component of the state vector in my
simulation. Stars B and C formed a new binary and ejected A for values of
$x_{14}$ between 20.3 and 21.2. For $x_{14} = 21.5$, I found a trajectory where
A and C form a new binary and B is ejected. This trajectory is at least
topologically similar, if not topologically identical, to the trajectory that
Hut and Bahcall found. The trajectory is shown in Figure 3.

\begin{center}
\includegraphics[scale=0.6]{ps13_3f.png}
\\
Figure 3: Stellar collision similar to Hut and Bahcall's result.
\end{center}

Note that the plot is still in a space fixed reference frame and so it's
difficult to tell whether or not one is looking at the same trajectory as Hut
and Bahcall had.

For $x_{14} = 21.8$, I found star B to stop in the center while stars A and C
were flung outwards at great speed, and for $x_{14} = 22.1$ I found a trajectory
where the stars diverge outward similar to the one shown in Figure 2.

\end{document}