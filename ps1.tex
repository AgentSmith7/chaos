\documentclass[12pt, letterpaper]{article}
\title{Problem Set 1: Logistic Map}
\author{Ken Sheedlo}
\usepackage[pdftex]{graphicx}
\usepackage{fullpage}
\usepackage{verbatim}
\usepackage{amsfonts}
\usepackage{caption}
\usepackage{subcaption}

\begin{document}
\maketitle{}

\section*{Fixed Point Behavior}

For $R < 3$, the logistic map converges to a fixed point. In the time domain,
the plot approaches a horizontal line from both sides. In the first return map
domain, points on the plot approach the line at $x_{n+1} = x_n$. Figure 1 shows
the logistic map in the time domain and first return map space. I used $R = 2.9, 
x_0 = 0.2$. 

\begin{minipage}[t]{0.5\textwidth}
\begin{center}
\includegraphics[scale=0.55]{log_fixed_time_cropped.png}
\end{center}
\end{minipage}
\begin{minipage}[t]{0.5\textwidth}
\begin{center}
\includegraphics[scale=0.55]{log_fixed_ret1_cropped.png}
\end{center}
\end{minipage}
\begin{center}
Figure 1: Fixed point behavior in the logistic map system.
\\
\end{center}

This demonstrates the behavior of the logistic map for $1 < R < 3$, which tends
toward a constant steady state.

\section*{Oscillating Behavior}

For $3 \leq R < 4$, the logistic map approaches a condition where it oscillates 
between two or more states. Plotting these reveals some interesting dynamics. 
Figure 2 shows the map in the time domain as well as the first and second return
map spaces using $R = 3.0, x_0 = 0.2$.

\begin{minipage}[t]{0.5\textwidth}
\begin{center}
\includegraphics[scale=0.55]{log_osc_time_cropped.png}
\includegraphics[scale=0.55]{log_osc_ret2_cropped.png}
\end{center}
\end{minipage}
\begin{minipage}[t]{0.5\textwidth}
\begin{center}
\includegraphics[scale=0.55]{log_osc_ret1_cropped.png}
\end{center}
\end{minipage}
\begin{center}
Figure 2: Oscillating behavior in the logistic map system.
\\
\end{center}

In the time domain, we see points on the plot approaching each other but 
remaining some distance apart. They will not meet in the middle as they did in
the same plot above for $R = 2.9$. At $R = 3.0$, the map demonstrates
oscillating behavior and will not converge on a point. Interestingly, this means
that the first and second return map spaces do not converge on a point either. 
The first return map space traces out points along a parabola and eventually 
oscillates between two of those points. There is a gap between the points where
the map does not go. Similarly, the second return map traces out points along a
fourth degree polynomial, but it also oscillates between two points and leaves a
gap rather than converging to a point. Oscillating behavior is evident in each
one of the three spaces.

\section*{Chaotic Divergence}

For certain values of $R$ between 3 and 4, the map becomes chaotic. This can be
demonstrated by changing $x_0$ slightly and watching where the plot goes. 
Figure 3 demonstrates chaotic behavior in the logistic map for $R = 3.9$.

\begin{center}
\includegraphics[scale=0.6]{log_chaotic_offset.png}
\\
Figure 3: Chaotic behavior.
\\
\end{center}

This is chaotic behavior because although the inputs varied only slightly, the
outcomes differed wildly. Of particular interest are the first ten or so data
points where the lines are almost exactly parallel before they diverge.

\section*{Further Analysis}

Using my program to evaluate various interesting $R$, I could not find evidence 
for chaos or any interesting structure at $R = 2$ or $R = 3.3$. At $R = 3.6$, I 
found chaotic behavior as well as an interesting oscillatory structure. It 
appears to be oscillating through a series of four states. At $R = 3.83$, I 
found more interesting oscillatory structure as well as evidence for chaos. The
sequences for $x_0 = 0.2$ and $x_0 = 0.2001$ diverged chaotically at first, but
converged back on the same steady state. At $R > 4$, the map quickly converges
to 0 and does not show evidence for chaotic behavior.

At $R = 2.5$, the system quickly converges to a fixed point at about 0.6
regardless of $x_0$. The fixed point is an attractor, and we say that the set of
initial conditions that converges to the attractor is a basin of attraction.

\end{document}
