\documentclass[12pt, letterpaper]{article}
\title{Problem Set 3: Fractals}
\author{Ken Sheedlo}
\usepackage[pdftex]{graphicx}
\usepackage{fullpage}
\usepackage{verbatim}
\usepackage{amsfonts}
\usepackage{caption}
\usepackage{subcaption}

\begin{document}
\maketitle{}

\section*{Newton's Method Fractal}

I went to the Google Labs site and explored the fractal for Newton's Method on
$x^3 - 1 = 0$. Figure 1 shows a plot of the fractal zoomed in several levels 
deep.

\begin{center}
\includegraphics[scale=0.4]{google_newton_fractal.png}
\\
Figure 1: Fractal structure generated by Newton's Method.
\end{center}

The plot clearly shows self-similar fractal structure. The teardrop shapes are 
bordered by smaller, self-similar teardrop shapes. At several points, they seem
to converge onto a point. This is very interesting and a good representative
fractal structure.

\section*{Capacity Dimension}

We wish to compute the capacity dimension for a middle-sixth-removed Cantor set.
Note that removing the middle sixth of a unit line segment leaves two smaller
segments, each of length $\frac{5}{12}$. Table 2 breaks down the number of 
$\varepsilon$-radius balls needed at each iteration of removing the middle 
sixth.

\begin{center}
\begin{tabular}{c | c | c}
$n$ & $N(\varepsilon)$ & $\varepsilon$ \\ \hline
1 & 1 & 1 \\
2 & 2 & $\frac{5}{12}$ \\
3 & 4 & $\frac{25}{144}$ \\
4 & 8 & $\frac{125}{1728}$ \\
\ldots & \ldots & \ldots \\
$n$ & $2^{n-1}$ & $(\frac{5}{12})^{n-1}$
\end{tabular}
\\
\vspace{1.0em}
Table 2: $N(\varepsilon)$ for a middle-sixth-removed Cantor set.
\end{center}

The capacity dimension is defined as 

\begin{equation}
d_c = \lim_{\varepsilon \to 0} \frac{\log(N(\varepsilon))}{\log(\frac{1}{\varepsilon})}
\end{equation}

We compute the capacity dimension:

\begin{equation}
d_c = \lim_{n \to \infty} \frac{\log(2^{n-1})}{\log((\frac{12}{5})^{n-1})}
\end{equation}

\begin{equation}
d_c = \frac{\log 2}{\log(\frac{12}{5})} = 0.79
\end{equation}

\section*{Fractal Tree}

I implemented a program to draw the fractal tree described in the assignment and
found that 13 iterations were necessary to make the differences between 
successive segments indistinguishable. The results are shown in Figure 2.

\begin{center}
\includegraphics[scale=0.6]{fractaltree.png}
\\
Figure 2: Self-similar fractal tree with length ratio 0.6.
\end{center}

As the segment length ratio decreases to 0.5 and lower, the tree appears to thin
out. More space clears out between the branches. As the ratio increases past 
0.707, the branches begin to overlap. In these plots, the tree appears more like
a noisy grid. One interesting implication is that for values of the length ratio
greater that 0.707, many points inside the volume of the tree could be reached
by branching down two or more different paths. For smaller length ratios, each 
point in the volume of the tree corresponds uniquely to a branch path.

Adapting the program to change segment length ratios and different angles
between the left and right branches yields interesting results. Figure 3 shows a
plot of the fractal tree with left segment ratio 0.7, right segment ratio 0.65, 
left branching angle 60 degrees, and right branching angle 40 degrees.

\begin{center}
\includegraphics[scale=0.6]{fractaltree_skewed.png}
\\
Figure 3: Rotated fractal tree with tuned parameters.
\end{center}

Figure 3 clearly demonstrates that changing segment length ratios and branching
angles changes the shape of the tree, but preserves self-similarity. We can make
some neat fractal trees by playing around with the parameters. Figures 4 and 5 
show a pair of representative fractal tree plots.

\begin{center}
\includegraphics[scale=0.6]{fractaltree_skew2.png}
\\
Figure 4: Representative plot of a fractal tree.
\end{center}

Figure 4 was generated with right and left segment ratios set to 0.6, left 
branching angle 30 degrees, and right branching angle 95 degrees. 

\begin{center}
\includegraphics[scale=0.6]{fractaltree_skew3.png}
\\
Figure 5: Another representative fractal tree plot.
\end{center}

Figure 5 was generated with left segment ratio 0.55, right segment ratio 0.7,
left branching angle 85 degrees, and right branching angle 5 degrees.

\end{document}