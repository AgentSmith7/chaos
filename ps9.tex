\documentclass[12pt, letterpaper]{article}
\title{Problem Set 9: Lyapunov Exponents}
\author{Ken Sheedlo}
\usepackage[pdftex]{graphicx}
\usepackage{fullpage}
\usepackage{verbatim}
\usepackage{amsfonts}
\usepackage{caption}
\usepackage{subcaption}

\begin{document}
\maketitle{}

\section*{2. Choosing the Delay Parameter}

To choose the delay parameter, I used the following \textsc{Tisean} command:

\vspace{1em}
\texttt{\$ mutual -D 180}
\vspace{1em}

For this assignment, I chose to write a light wrapper around the \textsc{Tisean}
package, which I call \texttt{tispy}.\footnote{This means that when I state that I used
a given \textsc{Tisean} command, I actually mean that \texttt{tispy} rendered my Python code
to produce that command. For instance, \texttt{\$ mutual -D 180} corresponds to 
\texttt{tispy.mutual('-D', 180, input=data)} where \texttt{data} is the data array 
read in from \texttt{data2.first250sec}.} My \texttt{tispy} library takes a Numpy 
\texttt{ndarray} object, passes it to \textsc{Tisean} and returns the result
back as a Numpy \texttt{ndarray}. Since I had \texttt{tispy} handle input and 
output for me, I did not need to pass \textsc{Tisean} a data file or specify an 
output file.

I found the first minimum of the mutual information plot at $\tau=155$. The 
results are shown in Figure 1.

\begin{center}
\includegraphics[scale=0.6]{ps9_1.png}
\\
Figure 1: Choosing the delay parameter $\tau$ for \texttt{data2.first250sec}.
\end{center}

The data show that the delay parameter $\tau$ that we should use is 155 steps or
0.31 seconds. To make time easier to reason about, I will give $\tau$ in units of
steps for the remainder of the discussion unless otherwise specified. One step is
0.002 seconds when working with the \texttt{data2} set and 0.001 seconds with the
other data sets.

\section*{3. Choosing the Embedding Dimension}

Knowing that I should use $\tau=155$ as my delay parameter, I used the following
\textsc{Tisean} command to find the embedding dimension to use:

\vspace{1em}
\texttt{\$ false\_nearest -M 1,10 -d 155}
\vspace{1em}

Figure 2 shows the resulting output.

\begin{center}
\includegraphics[scale=0.6]{ps9_2.png}
\\
Figure 2: Choosing the embedding dimension $m$ for \texttt{data2.first250sec}.
\end{center}

The plot shows that we should choose an embedding dimension of 8, because 8 is 
the first embedding dimension tested with less than 10\% false nearest neighbors.

\section*{4. Delay Coordinate Embedding with TISEAN}

Using the \texttt{delay} tool included with \textsc{Tisean}, I embedded 
\texttt{data2.first250sec} using $\tau=155$ and $m=8$. I used the following
\textsc{Tisean} command to do the embedding:

\vspace{1em}
\texttt{\$ delay -d 155 -m 8}
\vspace{1em}

The results are shown in Figure 3.

\begin{center}
\includegraphics[scale=0.6]{ps9_3.png}
\\
Figure 3: Delay coordinate embedding using \textsc{Tisean}.
\end{center}

I plotted the zeroth element of the resulting state vector, $\theta(t)$, against
the second, \newline$\theta(t+2\tau) = \theta(t+0.62)$. The results strongly 
resemble Figure 2 from Problem Set 8, although they are much less dense. This is 
reasonable because in this plot we are embedding only the first section of the 
data set, where in Problem Set 8 we embedded the entire data set. Both plots 
loop and zigzag in wild trajectories and cover the space evenly with few large
empty spaces in between. If we used \textsc{Tisean} to embed the entire data
set, we would most likely get a similar result as from Problem Set 8.

\section*{5. Estimating the Lyapunov Exponent}

I used the following \textsc{Tisean} command to help me estimate $\lambda$ for
the \texttt{data2} set.

\vspace{1em}
\texttt{\$ lyap\_k -d 155 -m 8 -M 8}
\vspace{1em}

This generated the data shown in Figure 4. Figure 4 plots the number of iterations
of Kantz's algorithm against the logarithm of the stretching factor.

\begin{center}
\includegraphics[scale=0.6]{ps9_4.png}
\\
Figure 4: Calculating the Lyapunov exponent with data from \texttt{lyap\_k}.
\end{center}

To find the Lyapunov exponent, I took a linear regression of the initial sloped
section of each curve on Figure 4 before it flattens out. This section turned
out to be just the first two points on each curve. Using just two points to find
the line is generally a questionable numerical method, but in this case it gave
a reasonable result. I found $\lambda$ to be 0.496714. This is an appropriate 
result because \texttt{data2} is a chaotic system, so the first Lyapunov exponent
should be positive nonzero.

Next I tried varying $m$ to demonstrate the effects of changing $m$ on the 
computed $\lambda$. Table 5 shows the results.

\begin{center}
\begin{tabular}{c | c}
$m$ & $\lambda$ \\ \hline
4 & 0.516219 \\
6 & 0.495411 \\
7 & 0.497139 \\
8 & 0.496714 \\
9 & 0.511835 \\
10 & 0.544658 \\
12 & 0.601088
\end{tabular}
\\
\vspace{1em}
Table 5: Computed Lyapunov exponents for varying $m$.
\end{center}

Table 5 shows that the computed $\lambda$ is stable near $m=8$, but changes
for $m \geq 10$ and $m < 6$. This suggests that one could use an embedding
dimension of 6 and still get a decent approximation of $\lambda$. Using $m=8$ 
could be more conservative than is really necessary. 

\newpage
\section*{6. Lyapunov Exponents for a Synthetic Data Set}

I generated a 30,000 point run of RK4 on the Lorenz system with parameters
$a=16, r=45, b=4$. I used \texttt{mutual} to estimate $\tau=0.105$ seconds or
105 steps of RK4. Figure 6 shows the delay coordinate embedding for the $x$
coordinate of this data.

\begin{center}
\includegraphics[scale=0.6]{ps9_6.png}
\\
Figure 6: Delay coordinate embedding the Lorenz system.
\end{center}

This embedding clearly shows a chaotic attractor. The two characteristic loops
in the Lorenz system are visible. The embedding should be topologically identical
to the original plot and indeed we can see some of the topology in the embedding,
so this is reasonable. 

Figure 7 shows the results from running \texttt{lyap\_k} on the generated Lorenz
data set.

\begin{center}
\includegraphics[scale=0.6]{ps9_7.png}
\\
Figure 7: Calculating the Lyapunov exponent for the Lorenz system.
\end{center}

Using the data shown in Figure 7, I computed the Lyapunov exponent to be 0.002861
in this system. It is surprising that the Lyapunov exponent is so close to 0 in
a chaotic system. Intuitively, the Lorenz system should have a larger positive
first Lyapunov exponent. 

Integrating the variational system and using the final eigenvalues to find
$\lambda_i$, we find $\lambda_1=0.013334$, $\lambda_2=0.012097$, 
$\lambda_3=0.011929$. This is a numerical limit at $t=30$ seconds, and 
$\lambda_i << 1$, so it is reasonable to argue that $\lambda_i$ are small and 
positive as $t$ approaches $\infty$. This is consistent with our results from 
before. We can conclude that the Lyapunov exponent in the Lorenz system is close 
to 0.

\end{document}